\section*{Contributions}
\label{section:contribution}

The scientific contributions of my thesis are :
\begin{list}{ \arabic{point}.}{%
		\usecounter{point}%
		\setlength{\topsep}{5pt}%
		\setlength{\itemsep}{0pt}%
		\setlength{\parsep}{0pt}%
		\setlength{\labelwidth}{3.em}%
		\setlength{\leftmargin}{2em}%
		\setlength{\labelsep}{0.5em}%
	}
\item[$\bluesquare$] the design a novel real time walking pattern generator using nonlinear model predictive control (Chap.~\ref{chap:nmpc}),
%The input is a $SE(2)$ velocity $[\dot{x},\dot{y},\dot{\theta}]$. It optimizes simultaneously the position and orientation of the left and right foot respectively.
%The use of a nonlinear solver allow additional functionalities like local obstacle avoidance.
\item[$\bluesquare$] the implementation of two multicontact pattern generators using the centroidal dynamics. They both take the contact placements as input and produce dynamically stable center of mass (CoM) motion (Chap.~\ref{chap:multicontact}).
\end{list}

In the context of the \koroibot\ project several technical contributions are to be noted:
\begin{list}{ \arabic{point}.}{%
		\usecounter{point}%
		\setlength{\topsep}{5pt}%
		\setlength{\itemsep}{0pt}%
		\setlength{\parsep}{0pt}%
		\setlength{\labelwidth}{3.em}%
		\setlength{\leftmargin}{2em}%
		\setlength{\labelsep}{0.5em}%
	}
\item[$\bluesquare$] the integration of a hose manipulation by HRP-2. The robot pick the hose from the floor and pull it to a desired position without falling,
\item[$\bluesquare$] the inclusion of the one third power law as control law for humanoid walking,
\item[$\bluesquare$] the elaboration of transfer rules to use upper body human motions integrated with a walking pattern generator (Chap.~\ref{chap:primitives}),
\item[$\bluesquare$] the study of potential application for humanoid robots in an industrial environment (Chap.~\ref{chap:univworker}) including,
\begin{list}{ \arabic{point}.}{%
		\usecounter{point}%
		\setlength{\topsep}{5pt}%
		\setlength{\itemsep}{0pt}%
		\setlength{\parsep}{0pt}%
		\setlength{\labelwidth}{3.em}%
		\setlength{\leftmargin}{2em}%
		\setlength{\labelsep}{0.5em}%
	}
	\item[$\bluesquare$] the use of fast planning in order to make HRP-2 walking towards a desired position while avoiding obstacles,
	\item[$\bluesquare$] the integration of visual servoing to track a target position while walking,
  \item[$\bluesquare$] the use of the whole body motion generator (Stack-of-Tasks) to evaluate the kinematic feasibility of screwing motions,
	\item[$\bluesquare$] the design of feet 3D trajectory to allow the robot to climb stairs or go across stepping stones and beams,  
\end{list}
\item[$\bluesquare$] the implementation of the dynamic filter on the LAAS-CNRS HRP-2 (An.~\ref{an:dyn:filter}),
\end{list}
