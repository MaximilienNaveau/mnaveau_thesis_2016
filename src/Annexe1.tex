\ifdefined\included
\else
\documentclass[a4paper,11pt,twoside]{StyleThese}
\include{formatAndDefs}
\sloppy
\begin{document}
\setcounter{chapter}{3} %% Num�ro du chapitre pr�c�dent ;)
\dominitoc
\dominilof
\faketableofcontents
\fakelistoffigures
\fi

\chapter{Annexe}
\label{an:dyn:filter}

\cleardoublepage

\section{The dynamic filter}

In this annexe we will present in details the dynamic filter algorithm.
This work has been published in \cite{naveau:ichr:2014}.
First of all we will briefly present the algorithm in the context of humanoid walking.
Then we will explicit technical details on the implementation.
Finally we will show that the walk of humanoid robot on flat ground is improved.

%
%The interest of the algorithm is that it re-introduce neglected dynamics inside the system. 
%, \softmetapod, a novel C++ library computing efficiently dynamic
%algorithms is presented. It uses template-programming techniques together with code-generation.
%The achieved performances shows a clear advantage over the state-of-the art dynamic library \softrbdl.
%The advantage of this library is that it is open-source and does not rely on any external symbolic computational software.
%Additionnaly we show how it can help in current control problems for humanoid robots,
%and more specifically for dynamic filtering of walking gait trajectories.
%

\input{Annexe1/dynamicfilter}

\begin{frame}{Conclusion}
\framesubtitle{Main contributions :}
\begin{itemize}
  \item Improvement of the versatility and performance of the flat ground walking.
  \item Decrease of the locomotion energy consumption using multiple contact system .
  \item Application of the walking pattern generator for pulling a hose.
  \item Introduction of the two third power law in the humanoid robot locomotion.
  \item Fusion of human extracted motion primitives and optimal control for whole body motion generation.
  \item Applications integrated for Airbus aircraft industry.
\end{itemize}

\end{frame}

\begin{frame}{Conclusion}
\framesubtitle{Perspectives :}
\begin{itemize}
  \item Pursue the  work on generalized locomotion by determining 3D trajectories for end effector and proper kinematic constraints.
  
  [Herdt 2010]
  \item Increase the versatility of the walking by using mixed-integer optimization. [Deits 2014]
\end{itemize}
\end{frame}



\ifdefined\included
\else
\bibliographystyle{StyleThese}
\bibliography{16-thesis-mnaveau}
\end{document}
\fi
