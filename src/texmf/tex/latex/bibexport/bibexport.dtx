% \iffalse meta-comment
%
% Copyright
%<*driver>
\documentclass{ltxdoc}
\usepackage{doc}

\def\script#1{\texttt{#1}}
\def\ext#1{\texttt{.#1}}
\def\var#1{\texttt{\$#1}}
\def\entry#1{\texttt{@#1}}
\def\lang#1{\textsc{#1}}
\def\bst#1{\textsf{#1.bst}}
\def\BibTeX{Bib\TeX}

%%%\AtBeginDocument{\CodelineIndex\EnableCrossrefs}
%%%\AtEndDocument{\PrintIndex}
\begin{document} 
 \def\docdate{2016/03/02}
 \def\fileversion{3.02}
%%% Please also change date at the beginning of the script %%%
%%% (VERSION and VDATE below)                              %%%
 \DocInput{bibexport.dtx}
\end{document}
%</driver>
% 
% \fi
% \CheckSum{41}
%% \CharacterTable
%%  {Upper-case    \A\B\C\D\E\F\G\H\I\J\K\L\M\N\O\P\Q\R\S\T\U\V\W\X\Y\Z
%%   Lower-case    \a\b\c\d\e\f\g\h\i\j\k\l\m\n\o\p\q\r\s\t\u\v\w\x\y\z
%%   Digits        \0\1\2\3\4\5\6\7\8\9
%%   Exclamation   \!     Double quote  \"     Hash (number) \#
%%   Dollar        \$     Percent       \%     Ampersand     \&
%%   Acute accent  \'     Left paren    \(     Right paren   \)
%%   Asterisk      \*     Plus          \+     Comma         \,
%%   Minus         \-     Point         \.     Solidus       \/
%%   Colon         \:     Semicolon     \;     Less than     \<
%%   Equals        \=     Greater than  \>     Question mark \?
%%   Commercial at \@     Left bracket  \[     Backslash     \\
%%   Right bracket \]     Circumflex    \^     Underscore    \_
%%   Grave accent  \`     Left brace    \{     Vertical bar  \|
%%   Right brace   \}     Tilde         \~}
%%
%
% \title{The \script{bibexport.sh} script}
% \author{Nicolas Markey}
% \date{\docdate}
% \maketitle
% \StopEventually{}
%
% \begin{abstract}
% \script{bibexport.sh} is a small shell script, relying on \BibTeX, that
% extracts entries of one or several \ext{bib} file(s). It~will expand
% abbreviations and cross-references, except standard month and journal
% abbreviations. The output is indented as neatly as possible, yielding a
% readable \ext{bib} file even if the original file is~not.
% \end{abstract}
%
%
% \section{Exporting \ext{bib} files}
%
% \subsection{Why and how?}
%
% \BibTeX{} aims at allowing for the use of one single \ext{bib} file,
% containing many entries, from which \BibTeX{} extracts only the
% \verb+\cite+d ones. When sending a document to someone else, this
% requires either sending the whole file, or extracting the \verb+\cite+d 
% entries from the \ext{bib} file.
%
% \BibTeX{} also has a mechanism for using abbreviations and
% cross-references. When extracting entries of a large \ext{bib} file,
% it can be interesting to develop those abbreviations, in order to get a
% clean, self-contained \ext{bib} file. Also, it may be useful to develop
% cross-references in a \ext{bib} file, independently of any document.
%
% \medskip
% \script{bibexport} can either extract entries that are cited in a document,
% or all the entries of one or several \ext{bib} files. It will always develop
% cross-references and abreviations, except standard abbreviations for months or
% some journals, that are defined in standard \BibTeX{} styles. This script
% uses \BibTeX{}. This has both pros and cons:
% \begin{itemize}
%\item[$+$] it is very simple. Basicaly, the script simply calls \BibTeX{}, and the
% \ext{bst} file just outputs the name and the content of each field. 
%\item[$+$] since it uses \BibTeX, we are sure that it will handle everything
% "properly", \emph{i.e.} in the same way as they will be handled when cited in
% a \LaTeX{} document;
%\item[$=$] \BibTeX{} has some strict limitations (especially "no more than 78 consecutive
% non-space characters") that we must be aware~of. On~the other hand, any such
% problem occuring within the script would also occur when compiling a
% document;
%\item[$-$] abbreviations and cross-references will \emph{always} be
% developped. It~could be argued that this is also a positive point, but
% having the choice would be better.
%\item[$-$] Many people seem to find \BibTeX's internal language clumsy, and thus
% the script could be difficult to adapt to special needs. However, this is
% not \emph{that} difficult, as will be explained later~on. In the present case, 
% adding more fields to be exported is quite easy.
% \end{itemize}
%
% \subsection{Related scripts}
%
% Several other tools exist for achieving this task:
% \begin{itemize}
%\item \verb+aux2bib+, written by Ralf Treinen, relies on \verb+bib2bib+,
% which is a \lang{CaML} program for selecting some entries in one or several \ext{bib}
% files. It does not expand anything, but includes all the necessary
% definitions and entries. 
%\item \verb+bibextract.sh+, by Nelson Beebe. This script uses \lang{awk} for
% extracting some entries out of a \ext{bib} file. It~is said not to be
% compliant with cross-references. 
%\item \bst{subset}, by David Kotz. \bst{export} develops the same ideas (but
% I discovered that only later~on). \bst{subset} does not handle
% \entry{preamble}, neither does it "protect" standard abbreviations. 
% \end{itemize}
%
% \subsection{Some examples}
%
% \begin{itemize}
%\item extracting \verb+\cite+d references of a document, also including
% cross-references:
%\begin{center}
% \ttfamily bibexport.sh -o {\itshape <result>}.bib {\itshape <file>}.aux
%\end{center}
%\item extracting \verb+\cite+d references of a document, without crossrefs,
% and using a special \ext{bst} file:
%\begin{center}
% \ttfamily bibexport.sh -b {\itshape <style>}.bst -o {\itshape <result>}.bib {\itshape <file>}.aux
%\end{center}
%\item export all the entries of two \ext{bib} files (including crossrefed entries):
%\begin{center}
% \ttfamily bibexport.sh -a -o {\itshape <result>}.bib {\itshape <file1>}.bib {\itshape <file2>}.bib
%\end{center}
%\item export all the entries of two \ext{bib} files (without crossrefs):
%\begin{center}
% \ttfamily bibexport.sh -a -n -o {\itshape <result>}.bib {\itshape <file1>}.bib
% {\itshape <file2>}.bib 
%\end{center}
% In fact, the only difference between this and the previous one is that
% \texttt{crossref} field will be filtered out at the end of the script. 
%\item export all the entries of two \ext{bib} files, using an extra file
% containing cross-referenced entries (which should not be included):
%\begin{center}
% \ttfamily bibexport.sh -a -e {\itshape <crossref>}.bib -n -o {\itshape
% <result>}.bib \textbackslash\par {\itshape <file1>}.bib {\itshape <file2>}.bib 
%\end{center}
%\end{itemize}
%
% \subsection{Exporting extra fields}
%
% By~default, \script{bibexport.sh} exports only "standard" fields
% (those defined and used in \bst{plain}), as~well as a few
% others. It~is very easy to modify it in order to export other fields:
% it~suffices to modify \bst{export} as follows:
% \begin{itemize}
% \item in the \texttt{ENTRY} list, add the name of the field you would
%   like to export. Notice that \texttt{ENTRY} takes three space-separated
%   lists as arguments; you must add extra fields in the first argument
%   (actually, the last two are empty).
% \item in the function \texttt{entry.export.extra}, add a line of the
% form
% \begin{center}
% \ttfamily
% "myfield" myfield field.export
% \end{center}
% where \texttt{myfield} is the name of the extra field you want to
% export.
% \end{itemize}
%
% \subsection*{Acknowledgements}
% I thank \'Eric Colin de Verdi\`ere, Richard Mathar, Harald Hanche-Olsen 
% and Damien Pollet for suggesting several improvements or corrections.
%
% \section{The code}
% \subsection{The shell script}
% \subsubsection{Initialization}
%
%\begin{macro}{checkversion}
% We check that the \ext{bst} files have the correct version number:
%    \begin{macrocode}
%<*script>
function checkversion()
{
  kpsewhich expcites.bst > /dev/null || 
    echo "-----------
--Warning-- file expcites.bst not found.
-----------"
  grep -q $VDATE `kpsewhich expkeys.bst` || 
    echo "-----------
--Warning-- the version of the .bst files does not match with that of this script.
-----------"
}
%</script>
%    \end{macrocode}
%\end{macro}
%
%\begin{macro}{usage}
% We first define how the script should be used:
%    \begin{macrocode}
%<*script>
function usage()
{
echo "bibexport: a tool to extract BibTeX entries out of .bib files.
usage: `basename $0` [-h|v|n|c|a|d|s|t] [-b|e|es|ec|o|r file] file...

Basic options: 
--------------
 -a, --all                  export the entire .bib files
 -o bib, --output-file bib  write output to file       [default: bibexport.bib]
 -t, --terse                operate silently
 -h, --help                 print this message and exit
 -v, --version              print version number and exit

Advanced options:
-----------------
 -b bst, --bst bst          specifies the .bst style file [default: export.bst]
 -c, --crossref             preserve crossref field               [default: no]
 -n, --no-crossref          remove crossref'd entries             [default: no]
 -e bib, --extra bib        extra .bib file to be used (crossrefs and strings)
 -es bib, --extras bib      extra .bib file to be used (for strings)
 -ec bib, --extrac bib      extra .bib file to be used (for crossrefs)
 -p, --preamble             write a preamble at beginning of output
 -r bib, --replace bib      replace .bib file(s) in the .aux file
 -d, --debug                create intermediate files but don't run BibTeX";
exit 0;
}
%</script>
%    \end{macrocode}
%\end{macro}
%
%\begin{macro}{opttoolate}
% We also have a function to warn if extra options are given after
% the names of input files, which is not allowed.
%    \begin{macrocode}
%<*script>
function opttoolate()
{
if [ ${TOOLATE} -ne 0 ]; then
    echo "No option is allowed after the input files";
    exit 0;
fi
}
%</script>
%    \end{macrocode}
%\end{macro}
%\begin{macro}{VERSION}
%\begin{macro}{VDATE}
%\begin{macro}{ALL}
%\begin{macro}{CREF}
%\begin{macro}{DEBUG}
%\begin{macro}{FILE}
%\begin{macro}{EXT}
%\begin{macro}{EXTRA}
%\begin{macro}{EXTRABIB}
%\begin{macro}{REPLACEBIB}
%\begin{macro}{NEWBIB}
%\begin{macro}{SPACE}
%\begin{macro}{BST}
%\begin{macro}{TERSE}
%\begin{macro}{BANNER}
%\begin{macro}{ARGS}
%\begin{macro}{TOOLATE}
% We define the default value of some variables:
% \begin{itemize}
% \item \var{VERSION}: the version number;
% \item \var{VDATE}: the release date;
% \item \var{ALL}: a flag indicating that all entries of the given (\ext{bib})
%   file are to be exported;
% \item \var{CREF}: the value of \verb+-min-crossrefs+;
% \item \var{FILE}: the input file(s);
% \item \var{EXT}: the extension (\ext{aux} or \ext{bib}) of input files;
% \item \var{EXTRA}: list of possible extra \ext{bib} files without extension;
% \item \var{EXTRABIB}: list of possible extra \ext{bib} files with extension;
% \item \var{REPLACEBIB}: flag indicating that we will replace the \ext{bib}
%   file given in the \ext{aux} file with a new one;
% \item \var{NEWBIB}: new \ext{bib} file to replace that fiven in the
%   \ext{aux} file;
% \item \var{SPACE}: file name separator (can be \textvisiblespace, comma or empty);
% \item \var{BST}: the \ext{bst} file to be used;
% \item \var{TERSE}: run silently;
% \item \var{BANNER}: don't print the initial comment;
% \item \var{ARGS}: the list of aruments passed to \texttt{bibexport.sh};
% \item \var{TOOLATE}: options are not allowed once we have encountered the
%   first non-option argument.
% \item \var{DEBUG}: create intermediate files but do not run BibTeX.
% \end{itemize}
%    \begin{macrocode}
%<*script>
## Version number
VERSION="3.02";
## Release date
VDATE="2016/03/02";

# ALL is a flag set to 1 when '-a' is given
ALL="0";
# FILE will be the main input file(s) (.aux or .bib, depending on '-a')
FILE="";
# EXT is the extension of the input file(s) (.aux, or .bib if '-a')
EXT=".aux";
# EXTRA and EXTRABIB are two copies of the extra files ('-e'), used to 
# include crossref'd entries and @string's
EXTRA="";
EXTRABIB="";
# REPLACEBIB ('-r') is set to 1 when the \bibdata of the .aux input file 
# must be ignores (then '-e' must be used)
REPLACEBIB="0";
# NEWBIB will contain the argument given to -r
NEWBIB="";
# BST is the .bst file to be used (default to export.bst)
BST="export";
# TERSE will be set to '-terse' if '-t' is given
TERSE="";
# BANNER is used to turn on or off the preamble informations in the output
BANNER="false";
# CREF is the number of citations of crossrefs from which the crossref'd entry 
# must be included. 
CREF="0";

# SPACE will be either ' ' or ','
SPACE="";
# TOOLATE is used to prevent extra options after the main file
TOOLATE="0";
# DEBUG is used to create files but not run BibTeX.
DEBUG="";

ARGS=$@;
%</script>
%    \end{macrocode}
%\end{macro}
%\end{macro}
%\end{macro}
%\end{macro}
%\end{macro}
%\end{macro}
%\end{macro}
%\end{macro}
%\end{macro}
%\end{macro}
%\end{macro}
%\end{macro}
%\end{macro}
%\end{macro}
%\end{macro}
%\end{macro}
%\end{macro}
%
% \subsubsection{Handling arguments}
%
% If no argument have been supplied, we call \script{usage}. Otherwise, we
% check version number.
%
%    \begin{macrocode}
%<*script>
if [ $# -eq 0 ]; then
  usage;
fi
checkversion;
%</script>
%    \end{macrocode}
%
% Otherwise, we enter a \verb+while+-loop for handling the whole list of arguments:
%    \begin{macrocode}
%<*script>
while [ $# != 0 ]; do
    case $1 in
%</script>
%    \end{macrocode}
%\begin{itemize}
%\item \verb+-a+ or \verb+--all+: export all the bibliography. This
% means that we input \ext{bib} files. 
%    \begin{macrocode}
%<*script>
        -a|--all)
            ## - export all entries in the input file(s)
            ## - the input files are BibTeX files
            opttoolate;
            EXT=""; SPACE=""; ALL=1;
            shift ;;
%</script>
%    \end{macrocode}
%\item \verb+-b+ or \verb+--bst+: specifies the style file. It seems
% that \BibTeX{} does not like the \texttt{./\textit{style}.bst}
% syntax, and we have to handle that case separately.
%    \begin{macrocode}
%<*script>
        -b|--bst) 
            ## - specifies the .bst file to use (default to 'export.bst')
            opttoolate;
            if [ "`dirname $2`" = "." ]; then
                DOLLARTWO="`basename $2 .bst`";
            else
                DOLLARTWO="`dirname $2`/`basename $2 .bst`";
            fi
            BST="${DOLLARTWO}";
            shift 2;;
%</script>
%    \end{macrocode}
%\item \verb+-d+ or \verb+--debug+: only creates (and preserves) the 
% intermediate files. This can help finding problems with the script or 
% \ext{bst} files.
%    \begin{macrocode}
%<*script>
        -d|--debug)
            ## - debug mode: we create files but do not run bibtex
            ## - instead, we print what we would have done...
            opttoolate;
            DEBUG="echo";
            shift ;;
%</script>
%    \end{macrocode}
%\item \verb+-e+ or \verb+--extra+: when we want to export all the
% entries of a \ext{bib} file, we can specify an extra \ext{bib} file
% that would contain entries that we don't want to export, but that
% are needed, \emph{e.g.} for crossrefs.
%    \begin{macrocode}
%<*script>
        -e|--extra)
            ## - extra input files (containing crossrefs or strings)
            ## - they will be included twice: once before the main file(s) 
            ##   (for @string's), once after (for crossrefs). We fool BibTeX 
            ##   by naming the first one 'file.bib' and the second one 
            ##   'file.bib.bib', to avoid complains.
            opttoolate;
            if [ "`dirname $2`" = "." ]; then
                DOLLARTWO="`basename $2 .bib`";
            else
                DOLLARTWO="`dirname $2`/`basename $2 .bib`";
            fi
            EXTRA="${EXTRA}${DOLLARTWO},";
            EXTRABIB="${EXTRABIB},${DOLLARTWO}.bib";
            shift 2;;
%</script>
%    \end{macrocode}
%\item \verb+-es+ or \verb+--extras+: if, for some reason, including extra 
% files twice is not possible, this options provides a way of including extra
% \ext{bib} files only before the main \ext{bib} file(s).
%    \begin{macrocode}
%<*script>
        -es|--extras)
            ## - extra input files (containing strings)
            ## - will be included *before* the main files (hence not suitable
            ##   for crossrefs)
            opttoolate;
            if [ "`dirname $2`" = "." ]; then
                DOLLARTWO="`basename $2 .bib`";
            else
                DOLLARTWO="`dirname $2`/`basename $2 .bib`";
            fi
            EXTRA="${EXTRA}${DOLLARTWO},";
            shift 2;;
%</script>
%    \end{macrocode}
%\item \verb+-ec+ or \verb+--extrac+: similar to te previous one, but for 
% file(s) included after the main \ext{bib} file(s).
%    \begin{macrocode}
%<*script>
        -ec|--extrac)
            ## - extra input files (containing crossrefs)
            ## - will be included only *after* the main files (hence not
            ##   suitable for @string's)
            opttoolate;
            if [ "`dirname $2`" = "." ]; then
                DOLLARTWO="`basename $2 .bib`";
            else
                DOLLARTWO="`dirname $2`/`basename $2 .bib`";
            fi
            EXTRABIB="${EXTRABIB},${DOLLARTWO}.bib";
            shift 2;;
%</script>
%    \end{macrocode}
%\item \verb+-o+ or \verb+--output+: the name of the output file.
%    \begin{macrocode}
%<*script>
        -o|--output-file)
            ## - name of the output file
            ## - we force it to end with '.bib'
            opttoolate;
            if [ "`dirname $2`" = "." ]; then
                DOLLARTWO="`basename $2 .bib`";
            else
                DOLLARTWO="`dirname $2`/`basename $2 .bib`";
            fi
            OUTPUT="${DOLLARTWO}.bib";
            shift 2 ;;
%</script>
%    \end{macrocode}
%\item \verb+-c+ or \verb+--crossref+ (or others): this options means
% that we want crossrefs to be included. Note that for any entry,
% field inheritage will be performed.
%    \begin{macrocode}
%<*script>
        -c|--crossref|--crossrefs|--with-crossref|--with-crossrefs)
            ## - whether or not to preserve 'crossref' keys.
            ## - by default, they are removed, but crossref'd entries are 
            ##   included.
            ## - crossrefs are *always* expanded anyway.
            opttoolate;
            CREF="1" ;
            shift ;;
%</script>
%    \end{macrocode}
%\item \verb+-n+ or \verb+--no-crossref+: don't include crossref'ed
% entries. 
%    \begin{macrocode}
%<*script>
        -n|--no-crossref|--without-crossref|--no-crossrefs|--without-crossrefs)
            ## - to remove crossref'd entries (hence remove 'crossref' keys).
            opttoolate;
            CREF="20000" ;
            shift ;;
%</script>
%    \end{macrocode}
%\item \verb+-r+ or \verb+--replace+: this provides a way of replacing the 
% \ext{bib} files given by \cmd{bibdata} in the \ext{aux} file with (a) new 
% one(s).
%    \begin{macrocode}
%<*script>
        -r|--replace)
            ## - to replace the file(s) given in \bibdata in the .aux file with
            ##   (a) new one(s).
            opttoolate;
            REPLACEBIB="1";
            if [ "`dirname $2`" = "." ]; then
                DOLLARTWO="`basename $2 .bib`";
            else
                DOLLARTWO="`dirname $2`/`basename $2 .bib`";
            fi
            NEWBIB="${NEWBIB}${DOLLARTWO}.bib,";
            shift 2;;
%</script>
%    \end{macrocode}
%\item \verb+-v+ or \verb+--version+ for version number:
%    \begin{macrocode}
%<*script>
        -v|--version)
            echo "This is bibexport v${VERSION} (released ${VDATE})"; exit 0;;
%</script>
%    \end{macrocode}
%\item \verb+-p+ or \verb+--preamble+ for inserting some informations 
% at the beginning of the output file:
%    \begin{macrocode}
%<*script>
        -p|--preamble|--with-preamble)
            BANNER="true";
            shift ;;
%</script>
%    \end{macrocode}
%\item \verb+-t+ or \verb+--terse+ for asking \BibTeX{} to run silently:
%    \begin{macrocode}
%<*script>
        -t|--terse|--silent)
            TERSE=" -terse ";
            shift ;;
%</script>
%    \end{macrocode}
%\item other dash-options are erroneous (except \verb+-h+, but...):
%    \begin{macrocode}
%<*script>
        -*)
            usage;;
%</script>
%    \end{macrocode}
%\item there should only remain file names: we add those names to the
% list of files.
%    \begin{macrocode}
%<*script>
        *)
            ## - list of input files
            ## - we ensure that no extra option is given later...
            TOOLATE="1";
            if [ "`dirname $1`" = "." ]; then
                DOLLARONE="`basename $1 ${EXT}`";
            else
                DOLLARONE="`dirname $1`/`basename $1 ${EXT}`";
            fi
            FILE="${FILE}${SPACE}${DOLLARONE}${EXT}";
            if [ ${ALL} -eq 1 ]; then
                SPACE=",";
            else
                SPACE=" ";
            fi;
            shift;;
%</script>
%    \end{macrocode}
% \end{itemize}
% That's all folks:
%    \begin{macrocode}
%<*script>
    esac
done
%</script>
%    \end{macrocode}
% 
% \subsubsection{The core of the script}
%
% We first set the name of the result and intermediary files: 
%    \begin{macrocode}
%<*script>
FINALFILE=${OUTPUT};
if [ ! "${FINALFILE}" ]; then
    FINALFILE="bibexport.bib";
fi
TMPFILE="bibexp.`date +%s`";
%</script>
%    \end{macrocode}
%
% We then create the \ext{aux} file for the main run of \BibTeX. Note
% that this could call \BibTeX, with the \bst{expkeys} file, in the
% case where we want to export all entries of a \ext{bib} file but not
% crossrefs. Note how, in that case, we trick \BibTeX for inputing
% extra files twice: we include then with their short name first (with
% no extension), and then with the full name. We \emph{need} to do that,
% since \verb+string+ abbreviations must be defined first, while
% crossrefs must occur after having been referenced.
%
%    \begin{macrocode}
%<*script>
if [ -z "${EXT}" ]; then ## we export all entries
    if [ -z "${EXTRA}" ]; then ## we have no extra files
        cat > ${TMPFILE}.aux <<EOF
\citation{*}
\bibdata{${FILE}}
\bibstyle{${BST}}
EOF
    else ## we have extra files (e.g. for crossrefs) but want all entries from ${FILE}
         ## we first extract the keys to be used:
        cat > ${TMPFILE}.aux <<EOF
\citation{*}
\bibdata{${FILE}}
\bibstyle{expkeys}
EOF
        ## This run may generate errors. We redirect the output:
        bibtex -min-crossrefs=${CREF} -terse ${TMPFILE} >/dev/null 2>&1;
        mv -f ${TMPFILE}.bbl ${TMPFILE}.aux;
        ## and then prepare the .aux file for exporting:
        cat >> ${TMPFILE}.aux <<EOF
\bibdata{${EXTRA}${FILE}${EXTRABIB}}
\bibstyle{${BST}}
EOF
    fi
else ## we only export entries listed in the given .aux file:
  if [ ! "x${REPLACEBIB}" = "x1" ]; then
    cat ${FILE} | sed -e "s/bibstyle{.*}/bibstyle{${BST}}/" > ${TMPFILE}.aux;
  else
    cat ${FILE} | sed -e "s/bibstyle{.*}/bibstyle{${BST}}/" \
      -e "s/bibdata{.*}/bibdata{${EXTRA}${NEWBIB%,}${EXTRABIB}}/" > ${TMPFILE}.aux;
  fi
fi
%</script>
%    \end{macrocode}
%
% This was the hard part. We now call \BibTeX, clean and rename the
% output file, and remove intermediary files:
%
%    \begin{macrocode}
%<*script>
if [ -z "$DEBUG" ]; then
    bibtex -min-crossrefs=${CREF} ${TERSE} ${TMPFILE};
    if [ -e ${FINALFILE} ]; then
        mv ${FINALFILE} ${FINALFILE}-save-`date "+%Y.%m.%d:%H.%M.%S"`
    fi
    echo "" > ${FINALFILE}
else
    echo "bibtex -min-crossrefs=${CREF} ${TERSE} ${TMPFILE};"
fi
if [ ! "${BANNER}" = "false" ]; then
    ## list of cited entries
    if [ -z "$DEBUG" ]; then
        sed -i -e "s/\\\bibstyle{.*}/\\\bibstyle{expcites}/" ${TMPFILE}.aux
        mv ${TMPFILE}.aux ${TMPFILE}-cites.aux
        bibtex -terse -min-crossrefs=${CREF} ${TMPFILE}-cites
        echo -ne "@comment{generated using bibexport:\n" >> ${FINALFILE};
        echo -ne "  creation date:\t`date +\"%c\"`\n" >> ${FINALFILE};
        echo -ne "  command:\t\t`basename $0` ${ARGS}\n" >> ${FINALFILE};
        if [ -z "${EXT}" ]; then
            echo -ne "  source files:\t\t${FILETAB}\t\t\t${EXTRABIBTAB}\n" >> ${FINALFILE}; \
                fi
        cat ${TMPFILE}-cites.bbl >> ${FINALFILE};
        echo -ne "  bibexport-version:\tv${VERSION} (${VDATE})\n" >> ${FINALFILE};
        echo -ne "  bibexport-maintainer:\tmarkey(at)lsv.ens-cachan.fr\n" >> ${FINALFILE};
        sed -i -e "s/}/)/g" ${FINALFILE};
        echo -n -e "}\n\n\n" >> ${FINALFILE};
        rm -f ${TMPFILE}-cites.bbl ${TMPFILE}-cites.aux ${TMPFILE}-cites.blg
    fi
fi
if [ ${CREF} -ne 1 ]; then
    if [ -z "$DEBUG" ]; then
        egrep -iv '^ *crossref *= *[^,]+,?$' \
            ${TMPFILE}.bbl >> ${FINALFILE};
    else
        echo "egrep -iv '^ *crossref *= *[^,]+,?$' ${TMPFILE}.bbl >> ${FINALFILE};"
    fi
else
    if [ -z "$DEBUG" ]; then
        cat ${TMPFILE}.bbl >> ${FINALFILE};
    else
        echo "cat ${TMPFILE}.bbl >> ${FINALFILE};"
    fi
fi
if [ -z "$DEBUG" ]; then
    rm -f ${TMPFILE}.bbl ${TMPFILE}.aux ${TMPFILE}.blg;
else
    echo "rm -f ${TMPFILE}.bbl ${TMPFILE}.aux ${TMPFILE}.blg";
fi
%</script>
%    \end{macrocode}
%
%
% \subsection{The \bst{expkeys} file}
%
% The only role of that file is to export the list of entries to be
% exported. It is used when we export all the entries of \ext{bib}
% files, except those of \emph{extra} \ext{bib} files. 
% Thus:
%    \begin{macrocode}
%<*expkeys>
ENTRY{}{}{}
READ
FUNCTION{export.key}
{
  "\citation{" cite$ "}" * * write$ newline$
}
ITERATE{export.key}
%</expkeys>
%    \end{macrocode}
%
% \subsection{The \bst{expcites} file}
%
% This file is used for exporting and formating the list of \verb+\cite+d entries.
% We begin with some parameters defining the margins
% \subsubsection{Some configuration values}
% \begin{macro}{left.width}
% \begin{macro}{right.width}
% \begin{macro}{url.right.width}
% \begin{macro}{left.short.width}
% \begin{macro}{right.short.width}
% \begin{macro}{left.delim}
% \begin{macro}{right.delim}
%    \begin{macrocode}
%<*expcites>
FUNCTION{left.width}{#23}
FUNCTION{right.width}{#55}
FUNCTION{url.right.width}{#61}
FUNCTION{left.short.width}{#10} %% for @preamble
FUNCTION{right.long.width}{#63}
FUNCTION{left.delim}{quote$}
FUNCTION{right.delim}{quote$}
%</expcites>
%    \end{macrocode}
% \end{macro}
% \end{macro}
% \end{macro}
% \end{macro}
% \end{macro}
% \end{macro}
% \end{macro}
%
% \subsubsection{Entries}
% We only want to export \verb+\cite+d keys, so we won't use any field.
% \begin{macro}{ENTRY}
%    \begin{macrocode}
%<*expcites>
ENTRY{dummy}{}{}
%</expcites>
%    \end{macrocode}
% \end{macro}
%
% \subsubsection{Basic functions}
%
%\begin{macro}{or}
%\begin{macro}{and}
%\begin{macro}{not}
%    \begin{macrocode}
%<*expcites>
FUNCTION{not}
{
    {#0}
    {#1}
  if$
}
FUNCTION{and}
{
    'skip$
    {pop$ #0}
  if$
}
FUNCTION{or}
{
    {pop$ #1}
    'skip$
  if$
}
%</expcites>
%    \end{macrocode}
% \end{macro}
% \end{macro}
% \end{macro}
%
% \subsubsection{Splitting strings}
%
% We design functions for splitting strings, so that the final
% \ext{bib} file will be cleanly indented. 
% \begin{macro}{space.complete}
% \begin{macro}{split.string}
%    \begin{macrocode}
%<*expcites>
INTEGERS{left.length right.length}
STRINGS{ s t }
INTEGERS{bool}
FUNCTION{space.complete}
{
  'left.length :=
  duplicate$ text.length$ left.length swap$ -
  {duplicate$ #0 >}
    {
      swap$ " " * swap$ #1 -
    }
  while$
  pop$
}
FUNCTION{split.string}
{
  'right.length :=
  duplicate$ right.length #1 + #1 substring$ "" =
    {""}
    {
      's :=
      right.length
      {duplicate$ duplicate$ s swap$ #1 substring$ " " = not and}
        {#1 -}
      while$
      duplicate$ #2 <
        {
          pop$ "    " s * ""
        }
        {
          duplicate$ s swap$ #1 swap$ substring$
          swap$
          s swap$ global.max$ substring$
        }
      if$
    }
  if$
}
%</expcites>
%    \end{macrocode}
% \end{macro}
% \end{macro}
%
% \subsubsection{Exporting cited entries}
% 
% Now we initialize, and export \verb+\cite+d entries.
%\begin{macro}{init.cited.keys}
%\begin{macro}{write.cited.keys}
%\begin{macro}{write.cited.keys.last}
%\begin{macro}{cited.keys}
%\begin{macro}{end.cited.keys}
%    \begin{macrocode}
%<*expcites>
FUNCTION{init.cited.keys}
{
  left.delim 's :=
  #0 'bool :=
}
FUNCTION{write.cited.keys}
{
  bool
    {"" left.width space.complete swap$}
    {"  cited keys: " left.width space.complete swap$
     #1 'bool :=}
  if$
  {duplicate$ text.length$ right.width >}
    {
      right.width split.string 't :=
      *
      write$ newline$
      "" left.width space.complete t
    }
  while$
  pop$ pop$ t
}
FUNCTION{write.cited.keys.last}
{
  bool
    {"" left.width space.complete swap$}
    {"  cited keys: " left.width space.complete swap$
     #1 'bool :=}
  if$
  {duplicate$ duplicate$ text.length$ #1 substring$ "," = not}
    {duplicate$ text.length$ #1 - #1 swap$ substring$}
  while$
  duplicate$ text.length$ #1 - #1 swap$ substring$
  right.delim *
  {duplicate$ "" = not}
    {
      right.width split.string 't :=
      *
      write$ newline$
      "" left.width space.complete t
    }
  while$
  pop$ pop$
}
FUNCTION{cited.keys}
{
  s cite$ ", " * *  's :=
  s text.length$ #4000 >
    {s write.cited.keys 's :=}
    'skip$
  if$
}
FUNCTION{end.cited.keys}
{
  s write.cited.keys.last
}
%</expcites>
%    \end{macrocode}
% \end{macro}
% \end{macro}
% \end{macro}
% \end{macro}
% \end{macro}
%
% \subsubsection{Now, we export...}
%
% We now export everything...
%    \begin{macrocode}
%<*expcites>
FUNCTION{article}{cited.keys}
FUNCTION{book}{cited.keys}
FUNCTION{booklet}{cited.keys}
FUNCTION{conference}{cited.keys}
FUNCTION{habthesis}{cited.keys}
FUNCTION{inbook}{cited.keys}
FUNCTION{incollection}{cited.keys}
FUNCTION{inproceedings}{cited.keys}
FUNCTION{journals}{cited.keys}
FUNCTION{manual}{cited.keys}
FUNCTION{mastersthesis}{cited.keys}
FUNCTION{misc}{cited.keys}
FUNCTION{phdthesis}{cited.keys}
FUNCTION{proceedings}{cited.keys}
FUNCTION{techreport}{cited.keys}
FUNCTION{unpublished}{cited.keys}
READ
EXECUTE{init.cited.keys}
ITERATE{cited.keys}
EXECUTE{end.cited.keys}
%</expcites>
%    \end{macrocode}
%
% \subsection{The \bst{export} file}
%
% \subsubsection{Some configuration values}
% \begin{macro}{left.width}
% \begin{macro}{right.width}
% \begin{macro}{url.right.width}
% \begin{macro}{left.short.width}
% \begin{macro}{right.short.width}
% \begin{macro}{left.delim}
% \begin{macro}{right.delim}
% We define here the indentation values, and the field
% delimiters. \emph{short} width are used for \entry{preamble}.
%    \begin{macrocode}
%<*export>
FUNCTION{left.width}{#18}
FUNCTION{right.width}{#55}
FUNCTION{url.right.width}{#61}
FUNCTION{left.short.width}{#10} %% for @preamble
FUNCTION{right.long.width}{#63}
FUNCTION{left.delim}{"{"}
FUNCTION{right.delim}{"}"}
%FUNCTION{left.delim}{quote$}
%FUNCTION{right.delim}{quote$}
%</export>
%    \end{macrocode}
%\end{macro}
%\end{macro}
%\end{macro}
%\end{macro}
%\end{macro}
%\end{macro}
%\end{macro}
% \subsubsection{Entries}
% We use standard entries here. Of course, more entries could be added
% for special \ext{bib} files. Those extra entries will also have to
% be added in the main exporting function.
% \begin{macro}{ENTRY}
%    \begin{macrocode}
%<*export>
ENTRY{
% Standard fields:
    address
    author
    booktitle
    chapter
    edition
    editor
    howpublished
    institution
    journal
    key
    month
    note
    number
    organization
    pages
    publisher
    school
    series
    title
    type
    volume
    year
% Special (but still somewhat standard) fields (natbib, germbib, ...):
    abstract
    acronym
    annote
    biburl
    doi
    eid
    isbn
    issn
    language
    url
    urn
}{}{}
%</export>
%    \end{macrocode}
% \end{macro}
%
% \subsubsection{Basic functions}
%
% No comment.
%\begin{macro}{or}
%\begin{macro}{and}
%\begin{macro}{not}
%    \begin{macrocode}
%<*export>
FUNCTION{not}
{
    {#0}
    {#1}
  if$
}
FUNCTION{and}
{
    'skip$
    {pop$ #0}
  if$
}
FUNCTION{or}
{
    {pop$ #1}
    'skip$
  if$
}
%</export>
%    \end{macrocode}
%\end{macro}
%\end{macro}
%\end{macro}
% \subsubsection{Splitting strings}
%
% We design functions for splitting strings, so that the final
% \ext{bib} file will be cleanly indented. This is also crucial
% to avoid long URLs.
% \begin{macro}{space.complete}
% \begin{macro}{split.string}
% \begin{macro}{split.url}
% \begin{macro}{split.name}
%    \begin{macrocode}
%<*export>
INTEGERS{left.length right.length}
STRINGS{ s t }
FUNCTION{space.complete}
{
  'left.length :=
  duplicate$ text.length$ left.length swap$ -
  {duplicate$ #0 >}
    {
      swap$ " " * swap$ #1 -
    }
  while$
  pop$
}
FUNCTION{split.string}
{
  'right.length :=
  duplicate$ right.length #1 + #1 substring$ "" =
    {""}
    {
      's :=
      right.length
      {duplicate$ duplicate$ s swap$ #1 substring$ " " = not and}
        {#1 -}
      while$
      duplicate$ #2 <
        {
          pop$ "    " s * ""
        }
        {
          duplicate$ s swap$ #1 swap$ substring$
          swap$
          s swap$ global.max$ substring$
        }
      if$
    }
  if$
}
FUNCTION{split.url}
{
  'right.length :=
  duplicate$ right.length #1 + #1 substring$ "" =
    {""}
    {
      's :=
      right.length
      {duplicate$ duplicate$ s swap$ #1 substring$ 
        duplicate$ "/" = swap$ 
        duplicate$ "&" = swap$
        duplicate$ "?" = swap$
        duplicate$ "-" = swap$ 
                   ":" = or or or or not and}
        {#1 -}
      while$
      duplicate$ #2 <
        {
          pop$ "    " s * ""
        }
        {
          duplicate$ s swap$ #1 swap$ substring$
          swap$ #1 +
          s swap$ global.max$ substring$
        }
      if$
    }
  if$
}
FUNCTION{split.name}
{
  'right.length :=
  duplicate$ right.length #1 + #1 substring$ "" =
    {""}
    {
      's :=
      right.length
      {duplicate$ duplicate$ s swap$ #5 substring$ " and " = not and}
        {#1 -}
      while$
      duplicate$ #2 <
        {
          pop$ "  " s * ""
        }
        {
          #4 + duplicate$ s swap$ #1 swap$ substring$
          swap$
          s swap$ global.max$ substring$
        }
      if$
    }
  if$
}
%</export>
%    \end{macrocode}
% \end{macro}
% \end{macro}
% \end{macro}
% \end{macro}
% \subsubsection{Exporting fields}
% 
% Here, we have four exporting functions, since we also have to deal
% with abbreviations: 
%\begin{macro}{field.export}
%\begin{macro}{abbrv.export}
%\begin{macro}{name.export}
%\begin{macro}{url.export}
%    \begin{macrocode}
%<*export>
FUNCTION{field.export}
{
  duplicate$ missing$
    'skip$
    {
      left.delim swap$ * right.delim *
      swap$
      "  " swap$ * " = " * left.width space.complete
      swap$ "," *
      {duplicate$ "" = not}
        {
          right.width split.string 't :=
          *
          write$ newline$
          "" left.width space.complete t
        }
      while$
    }
  if$
  pop$ pop$
}
FUNCTION{abbrv.export}
{
  duplicate$ missing$
    'skip$
    {
      swap$
      "  " swap$ * " = " * left.width space.complete
      swap$ "," *
      {duplicate$ "" = not}
        {
          right.width split.string 't :=
          *
          write$ newline$
          "" left.width space.complete t
        }
      while$
    }
  if$
  pop$ pop$
}
FUNCTION{name.export}
{
  duplicate$ missing$
    'skip$
    {
      left.delim swap$ * right.delim *
      swap$
      "  " swap$ * " = " * left.width space.complete
      swap$ "," *
      {duplicate$ "" = not}
        {
          right.width split.name 't :=
          *
          write$ newline$
          "" left.width space.complete t
        }
      while$
    }
  if$
  pop$ pop$
}
FUNCTION{url.export}
{
  duplicate$ missing$
    'skip$
    {
      left.delim swap$ * right.delim *
      swap$
      "  " swap$ * " = " * left.width space.complete
      swap$ "," *
      {duplicate$ "" = not}
        {
          url.right.width split.url 't :=
          *
          write$ newline$
          "" left.width space.complete t
        }
      while$
    }
  if$
  pop$ pop$
}
%</export>
%    \end{macrocode}
%\end{macro}
%\end{macro}
%\end{macro}
%\end{macro}
% \subsubsection{Handling abbreviations}
%
% Abbreviations are difficult to deal with if we wish to still use
% them, since \BibTeX will expand them before we can do anything. All we
% can do is to define them in a special way, in order to be able to get
% back to the abbreviations later on. This is precisely what we do:
%\begin{macro}{jan-dec}
%\begin{macro}{acmcs-tcs}
%\begin{macro}{remove.exports.from.months}
%\begin{macro}{remove.export.from.journal}
%    \begin{macrocode}
%<*export>
MACRO{jan}{"export-jan"}
MACRO{feb}{"export-feb"}
MACRO{mar}{"export-mar"}
MACRO{apr}{"export-apr"}
MACRO{may}{"export-may"}
MACRO{jun}{"export-jun"}
MACRO{jul}{"export-jul"}
MACRO{aug}{"export-aug"}
MACRO{sep}{"export-sep"}
MACRO{oct}{"export-oct"}
MACRO{nov}{"export-nov"}
MACRO{dec}{"export-dec"}
MACRO{acmcs}{"export-acmcs"}
MACRO{acta}{"export-acta"}
MACRO{cacm}{"export-cacm"}
MACRO{ibmjrd}{"export-ibmjrd"}
MACRO{ibmsj}{"export-ibmsj"}
MACRO{ieeese}{"export-ieeese"}
MACRO{ieeetc}{"export-ieeetc"}
MACRO{ieeetcad}{"export-ieeetcad"}
MACRO{ipl}{"export-ipl"}
MACRO{jacm}{"export-jacm"}
MACRO{jcss}{"export-jcss"}
MACRO{scp}{"export-scp"}
MACRO{sicomp}{"export-sicomp"}
MACRO{tocs}{"export-tocs"}
MACRO{tods}{"export-tods"}
MACRO{tog}{"export-tog"}
MACRO{toms}{"export-toms"}
MACRO{toois}{"export-poois"}
MACRO{toplas}{"export-toplas"}
MACRO{tcs}{"export-tcs"}
INTEGERS{ intxt }
FUNCTION{remove.exports.from.months}
{
  #0 'intxt :=
  duplicate$ missing$
    'skip$
    {'t :=
    ""
    {t #1 #1 substring$ "" = not}
      {
      t #1 #7 substring$ "export-" =
        {intxt
           {right.delim * #0 'intxt :=}
           'skip$
         if$
         duplicate$ "" =
           'skip$
           {" # " *}
         if$
         t #8 #3 substring$ *
         t #11 global.max$ substring$ 't :=}
        {intxt
           'skip$
           {duplicate$ "" = 
              {}
              {" # " *}
            if$
            left.delim * #1 'intxt :=}
         if$
         t #1 #1 substring$ *
         t #2 global.max$ substring$ 't :=}
      if$
      }
    while$
    intxt
      {right.delim *}
      'skip$
    if$
    }
  if$
}
FUNCTION{remove.export.from.journals}
{
  duplicate$ missing$
    'skip$
    {
      duplicate$ #1 #7 substring$ "export-" =
        {#8 global.max$ substring$}
        {left.delim swap$
         right.delim * *}
      if$
    }
  if$
}
%</export>
%    \end{macrocode}
%\end{macro}
%\end{macro}
%\end{macro}
%\end{macro}
%
% \subsubsection{Now, we export...}
%
% We gather everything. This is were special fields must be added for
% being exported:
%\begin{macro}{entry.export.standard}
%\begin{macro}{entry.export.extra}
%\begin{macro}{entry.export}
%\begin{macro}{export}
%    \begin{macrocode}
%<*export>
FUNCTION{entry.export.standard}
{
  "address" address field.export
  "author" author name.export
  "booktitle" booktitle field.export
  "chapter" chapter field.export
  "crossref" crossref field.export
  "edition" edition field.export
  "editor" editor name.export
  "howpublished" howpublished field.export
  "institution" institution field.export
  "journal" journal remove.export.from.journals abbrv.export
  "key" key field.export
  "month" month remove.exports.from.months abbrv.export
  "note" note field.export
  "number" number field.export
  "organization" organization field.export
  "pages" pages field.export
  "publisher" publisher field.export
  "school" school field.export
  "series" series field.export
  "type" type field.export
  "title" title field.export
  "volume" volume field.export
  "year" year field.export
}
FUNCTION{entry.export.extra}
{
  "abstract" abstract field.export
  "acronym" acronym field.export
  "annote" annote field.export
  "biburl" biburl url.export
  "doi" doi field.export
  "eid" eid field.export
  "isbn" isbn field.export
  "issn" issn field.export
  "language" language field.export
  "url" url url.export
  "urn" urn url.export
}
FUNCTION{entry.export}
{
  entry.export.standard
  entry.export.extra
}
FUNCTION{export}
{
  "@" type$ * "{" * cite$ * "," * write$ newline$
  entry.export
  "}" write$ newline$ newline$
}
%</export>
%    \end{macrocode}
%\end{macro}
%\end{macro}
%\end{macro}
%\end{macro}
%
% \subsubsection{Miscellanea}
% We also have to handle preamble, and to define functions for each
% entry type (we won't use them but otherwise, \BibTeX would complain). 
%
%\begin{macro}{preamble}
%\begin{macro}{header}
%\begin{macro}{entries.headers}
%\begin{macro}{article-unpublished}
%    \begin{macrocode}
%<*export>
FUNCTION{preamble}
{
preamble$ duplicate$ "" =
  'pop$
  {
    ",-------------------." write$ newline$
    "|     PREAMBLE      |" write$ newline$
    "`-------------------'" write$ newline$ newline$
    "@preamble{ "  swap$
    quote$ swap$ * quote$ *
    {duplicate$ "" = not}
      {
        right.long.width split.string 't :=
        *
        write$ newline$
        "" left.short.width space.complete t
      }
    while$
    "}" write$ newline$ newline$
    pop$ pop$
  }
if$
}
FUNCTION{header}
{
%"** This file has been automatically generated by bibexport **" 
%write$ newline$
%"** See   http://www.lsv.ens-cachan.fr/~markey/bibla.php    **" 
%write$ newline$
%"** for more informations about bibexport.                  **" 
%write$ newline$
newline$
}
FUNCTION{entries.header}
{
preamble$ "" =
  'skip$
  {
    ",-------------------." write$ newline$
    "|  BIBTEX ENTRIES   |" write$ newline$
    "`-------------------'" write$ newline$ newline$
  }
if$
}
FUNCTION{article}{export}
FUNCTION{book}{export}
FUNCTION{booklet}{export}
FUNCTION{conference}{export}
FUNCTION{habthesis}{export}
FUNCTION{inbook}{export}
FUNCTION{incollection}{export}
FUNCTION{inproceedings}{export}
FUNCTION{journals}{export}
FUNCTION{manual}{export}
FUNCTION{mastersthesis}{export}
FUNCTION{misc}{export}
FUNCTION{phdthesis}{export}
FUNCTION{proceedings}{export}
FUNCTION{techreport}{export}
FUNCTION{unpublished}{export}
%</export>
%    \end{macrocode}
%\end{macro}
%\end{macro}
%\end{macro}
%\end{macro}
% \subsubsection{Main program}
%
% We now can execute and iterate those functions:
%
%    \begin{macrocode}
%<*export>
READ
EXECUTE{header}
EXECUTE{preamble}
EXECUTE{entries.header}
ITERATE{export}
%</export>
%    \end{macrocode}
%
% \Finale
\endinput
