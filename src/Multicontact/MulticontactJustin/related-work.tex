\section{Related work}
\label{sec:related}
Our CoM estimation approach is part of an active topic both in research on human motion and in humanoid robotics~\cite{Cotton2009itm}. For humanoids, the corrections on the CoM provided by forward kinematics is achieved mainly using various measurement systems~\cite{fallon2014drift} including force sensors~\cite{stephens2011state,Xinjilefu2012iros}. These solutions use mostly Kalman filtering techniques which is agnostic of the frequency domains of each signal. On the other hand, the CoM reconstruction has a longer history in the field of biomechanics~\cite{Eng1993gaitnposture}. Moreover, since few decades, force platforms were already considered for CoM position estimation~\cite{shimba1984estimation}. Most of the methods did not consider the fusion of Force sensors with direct kinematics reconstruction of the CoM~\cite{caron1997jbiomech, barbier2003estimation}. 

To our best knowledge, the closest published work to our method is the technique by Maus et al~\cite{maus2011combining}. The Kinematic CoM estimation was low-pass filtered and the double time-integral of forces was high-pass filtered, then the two signals were added. However, the use of non-complementary filtering removes the guarantee to obtain the totality of the initial signal with zero-phase-shift. Moreover, the double integration of acceleration is not a stable process and this method requires to reset regularly the integral to zero. Instead, our method works for arbitrary durations thanks to the stability of all our filters. Schepers et al~\cite{schepers2009biomen} developed the same approach as Maus et al, but with ZMP and force measurements. In addition to theoretical guarantees and integration stability issues, this method neglected the horizontal accelerations for the ZMP, which increases the approximation errors.
