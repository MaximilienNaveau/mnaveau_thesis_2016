%!TEX root = root.tex
\section{Motion Generation}
\label{sec:motion_generation}
%
%\begin{figure}
%\vspace*{0.3cm}
%\begin{center}
%\tikzstyle{MGstyle} =[text centered,draw=blue!50,fill=blue!20,thick,]
%\begin{tikzpicture}[font=\normalsize, node distance=1cm]
%  \node (CP)   at (0.0,0) [shape=rectangle,MGstyle,text width=2.5cm] {Contact Planner};
%  \node (OCP)  at (3.2,1.5) [MGstyle,text width=2.5cm] {Optimal Control Problem} edge [<-,bend right=45] (CP);
%  \node (EETG) at (3.2,-1.5) [MGstyle,text width=2.5cm] {End Effector Trajectory Generation} edge [<-,bend left=45] (CP);
%  \node (GIK)  at (6.4,0) [MGstyle,text width=2.5cm] {Task-Space Whole Body}
%  edge [<-,bend left=45] node [auto] {${\bf RF},{\bf LF},{\bf RH},{\bf LH}$}(EETG)
%  edge [<-,bend right=45] node [auto,swap] {${\bf c}$} (OCP);
%  \node (Q)    at (9,0)  {$\bf{q}$} edge [<-] (GIK);
%\end{tikzpicture}
%\end{center}
%\caption{Overview of the joint trajectories (${\bf q}$) generation process, with ${\bf c}$ the CoM trajectory, and ${\bf RF},{\bf LF},{\bf RH},{\bf LH}$ the end-effector trajectories.}
%\label{fig:overview_joint_trajectories}
%\end{figure}
%
The computation of the trajectories, depicted in Fig.~\ref{fig:overview_joint_trajectories} and needed for the realization of the motion, is a five step process.
(i) A set of contact points has to be found.
(ii) The trajectories of the center of mass is computed by the $OCP$.
(iii) The full trajectory of the end effectors are computed using B-splines.
(iv) The generalized inverse kinematics algorithm is used to compute the actuated joint trajectories of the robot.
(v) The robot perform the computed motion.



In the final process, a set of joint trajectories (${\bf q}$) is provided for the position-controlled humanoid robot HRP-2.
Note that these trajectories are dynamically consistent and that the contacts are realized according to the predefined schedule of the OCP formulation.
The change of angular momentum ${\bf \sigma}$ was of small effect for the stair climbing motion and was neglected.
In cases where ${\bf \sigma}$ affects the motion, it can be rejected by methods of resolved momentum control using the free limbs during the execution of the motion.

\subsection{Definition of the contact sequence}

For each contact the time interval and the contact state, active or not, are defined.
We also specify which body is in contact, the position of the contact, the friction coefficient, and the contact normal vector.
The contact sequence is a collection of such contact specifications.
Such a contact sequence is typically realized through a contact planner \cite{Keddar:iros:2014,tonneau2015isrr}.
An extension to this work would be to additionally optimize the position of the contacts on a given set of planar contact surfaces.

\subsection{End-effector trajectories}

In this chapter, the hand, the feet and the CoM trajectories are represented using B-splines.
In the frame of the Koroibot project, human hand and foot trajectories are studied using inverse optimal control.
%A model predictive control based on this work, will be used to optimize the end trajectory of the end effector.
In addition, meaningful optimization criteria minimized by the under-actuated part of the human dynamics are investigated and will be included in the cost function of problem~\eqref{eq:optimal_control_objective}.
At this stage of the project, this is still on-going work and the B-spline parameters are found using heuristics.
Although not in the scope of this chapter, other approaches are possible.
% A prior exploration of this space through PRM could be done, but the size of the space make this approach computationally difficult.
% We could also teach the robot by demonstration using dynamic motion primitives instead of B-Splines.
% The key part of B-splines is to avoid local minima and provide guiding trajectories which drive the robot away from its constraints (joint limits, self-collision) and from the environment constraints (obstacles to avoid).

\subsection{Whole-body generation}
The final whole-body motion is generated by applying the stack-of-tasks~(SoT) scheme implementing a generalized inverse kinematics~(GIK) as shown in \cite{mansard:icar:09}.
Given the CoM, the root orientation and end-effector trajectories the SoT framework computes the complete trajectory of all the DoFs of the system.
This is done by specifying tasks for the SoT.
Those tasks are defined as a simple proportional derivative (PD) controller tracking the corresponding reference trajectory.
The tasks are the following: a task tracking the CoM trajectory along the $x,y,z$ axes ($T_{Com}$), a task for each end-effector position and orientation specification ($T_{RH,LH,RF,LF}$), a task to control the orientation of the waist ($T_{W}$), and a task regulating the posture of the robot around a nominal posture ($T_{q}$).
The hierarchy of the tasks defined as the lexicographic order
%\begin{equation*}
$T_{Com} \prec T_{RH,LH,RF,LF} \prec T_{W}  \prec T_{q}$.
%\end{equation*}
%In the related SoT problem, the free variables are the joint accelerations $\ddot{\bf q}$.
%The GIK solver assumes that the contact forces are generated at the proper time instants.% but does not check this constraint explicitly.
The dynamical consistency of the solution with respect to the robot model and the contact forces is implicitly given by the properties of the CoM trajectory computed by the OCP.

The corresponding contact forces and joint torques are then reconstructed with a dynamic simulator.
For each time step, the contact forces are computed as the minimal forces corresponding to the joint trajectory ${\bf q}$, $\dot{\bf q}$, $\ddot{\bf q}$ and respecting the contact model by
$$ \min_{{\bf f}_1 \dots {\bf f}_M} || RNEA({\bf q},{\bf \dot q},{\bf \ddot q}) - \sum_i {\bf J}_i^T {\bf f}_i ||^2 $$
such that ${\bf f}_i \in {K}_i, \ i = 1..M $, where $RNEA$ stands for the {\it Recursive Newton Euler Algorithm}~\cite{luh1980line}. %, and computes the centroidal torques to be actuated by the contact forces.
The motion can be checked to be dynamically consistent if the residual is null for all time instants of the movement.
