%!TEX root = root.tex
\label{sec:pb_statement}

\subsection{Dynamic model and constraints}
\label{sec:dynmodel}

\subsubsection*{Inertia ellipsoid model}
The template model is composed by the number of considered contact points $M\in\N$ located at positions $\pf \in \mathbb{R}^3$, for $i=1\dots M$.
The force applied at $\pf$, denoted by $\ff = [ \; \ffx \; \ffy \; \ffz \; ]^{\top}$, is given in a local coordinate system, with the $z$-axis normal to the contact surface at $\pf$.
The component $\ffz$ is the normal force applied at contact point $\pf$ and $(\ffx, \ffy)$ is the tangential force applied at $\pf$.
In the context of the stair climbing motion, $M=3$, and the involved robot limbs are the right hand $p_{rh}$, the left foot~$p_{lf}$ and right foot~$p_{rf}$.

Following the ideas from \cite{Orin:autorob:2013,Nishiwaki:IJRR:2012}, we only consider the effect of the whole-body dynamics of a humanoid robot on the CoM. The centroidal dynamics is described as
\begin{equation}
  \label{eq:equilibrium_constraints}
  m
  \begin{bmatrix}
  \ddcom - {\bf g} \\
  \com \times (\ddcom - {\bf g})
  \end{bmatrix}
  +
  \begin{bmatrix}
  {\bf 0} \\
  \dot{\mathcal{L}}
  \end{bmatrix}
  =
  \begin{bmatrix}
  \sum_{i=1}^{M}
  \Qf \ff \\
  \sum_{i=1}^{M}
  \pf \times \Qf  \ff
\end{bmatrix}
\end{equation}
where $\com = [ c_x \; c_y \; c_z ]^{\top}\in \R^3$ is the CoM and $m\in\R$ is the mass of the humanoid multi-body system, $\ddcom = [ \ddot{c}_x \; \ddot{c}_y \; \ddot{c}_z ]^{\top}\in\R^3$ is the CoM acceleration, $\dot{\mathcal{L}}$ is the derivative of the robot angular momentum due to the angular speed of the robot body parts, and $\Qf \in SO(3)$ is the $3\times3$ rotation matrix that transforms forces ${\bf f}_i$ in the local contact coordinate system at $\pf$ into the global world coordinate system.

%\todo{This can be put into the introduction, but does not provide much here.}
%\emph{Hirukawa~et~al.}~\cite{Hirukawa:icra:2007} consider a set of contact points, the velocity of the feet, the hands, the free-flyer, an initial guess on the CoM, and a heuristic to distribute the amount of forces on the %contact points.%
%Knowing this, they proposed a pattern generator that satisfy the friction cones using so called resolved momentum control. This algorithm tracks an initial guess \todo{of what?} while creating a dynamically feasible motion for %the CoM.

In the following, the contribution to the variation of the angular momentum $\mathcal{L}$ is separated between the part caused by the general angular acceleration of the robot and the internal robot body movements $\dot{\bf\mathcal{L}} = {\bf I}_c \; \dot{\bf \omega} - {\bf\sigma}, $
where $ {\bf I}_c $ is the inertia matrix of the whole body considered as a single rigid body computed for the configuration at time $ t_0 $, $ \dot{\bf \omega} $ is the angular acceleration (3D) of the root frame attached to the waist body of the robot, and ${\bf\sigma}$ is a function of the whole-body configuration, velocity and acceleration~\cite{Orin:autorob:2013} that does not depend on ${\bf\omega}$.
%
Eq.~\ref{eq:equilibrium_constraints} is then rewritten in form of
\begin{equation}
  \label{eq:equilibrium_constraints_Ic}
   {\bf M}(\com)
   \begin{bmatrix}
     \ddcom \\ \dot{\bf \omega}
   \end{bmatrix}
   =   \sum_i {\bf J}_i^T \ff + {\bf S}^T {\bf\sigma} + {\bf b}(\com),
\end{equation}
% \begin{equation}
%   \label{eq:equilibrium_constraints_Ic}
%   \begin{bmatrix}
%     m (\ddcom - {\bf g}) \\
%     m (\com \times (\ddcom - {\bf g})) + {\bf I}_c \; \dot{\bf \omega}
%   \end{bmatrix}
%   =
%   \begin{bmatrix}
%     \sum_{i=1}^{M} {\bf Q}_i \ff \\
%     \sum_{i=1}^{M} \pf \times {\bf Q}_i  \ff
%   \end{bmatrix}
% \end{equation}
with
\begin{align}
  \label{eq:matrix_linear}
  {\bf M}(\com) &=
  \begin{bmatrix}
    m \mathbf{1}_{3} & {\bf 0}_{3 \times 3} \\
    m \com \times & {\bf I}_c
  \end{bmatrix}
  ,\
  &{\bf J}_i^T &=
  \begin{bmatrix}
    {\bf Q}_i \\
    {\bf p}_i\times {\bf Q}_i
  \end{bmatrix},
  %
  \\
  %
  {\bf S}^T &=
  \begin{bmatrix}
    {\bf 0}_{3} \\
    \mathbf{1}_{3}
  \end{bmatrix}
  ,\
  &{\bf b}(\com) &=
  \begin{bmatrix}
    m\ {\bf g} \\
    m \com \times {\bf g}
  \end{bmatrix},
\end{align}
where $\com\times\in\R^{3 \times 3}$ (resp. ${\bf p}_i \times$) is the skew matrix associated with vector $\com$ (resp. ${\bf p}_i$).

%\subsection*{Equilibrium constraints and robot dynamics}
The equilibrium constraints \eqref{eq:equilibrium_constraints_Ic} are rewritten according to \cite{boyd:tro:07} as non-linear first-order differential equation in form of
\begin{equation}
\label{eq:equilibrium_constraints_linear}
  \frac{d}{dt}
  \begin{bmatrix}
    \com \\
    \comr \\
    \dcom \\
    {\bf \omega}
  \end{bmatrix}
  =
  \begin{bmatrix}
    \dcom \\
    \dcomr \\
    ({\bf M}(\com))^{-1}\left(\sum_i {\bf J}_i^T \ff + {\bf S}^T \sigma + {\bf b}(\com) \right)
  \end{bmatrix}.
\end{equation}

\subsubsection*{Linear constraints}
The applied contact model treats the feet contacts as unilateral and the hand contact on the rail as bilateral.
This is achieved by defining proper bounds onto the applied contact forces given by
$$
\resizebox{\linewidth}{!}{% Resize table to fit within \linewidth horizontally
\parbox{\linewidth}{
\begin{align*}
    \begin{bmatrix}
        \begin{smallmatrix}
            -600\,N\\
            -600\,N\\
               0\,N
        \end{smallmatrix}
    \end{bmatrix}
    \leq
    \begin{matrix}
        {\bf f_{lf/rf}}
    \end{matrix}
    \leq
    \begin{bmatrix}
        \begin{smallmatrix}
        600\,N\\
        600\,N\\
        600\,N
        \end{smallmatrix}
    \end{bmatrix}%
    ,\;\;\;\;\;\;
    \begin{bmatrix}
        \begin{smallmatrix}
        -150\,N\\
        -150\,N\\
        -150\,N%
        \end{smallmatrix}
    \end{bmatrix}%
    \leq
    \begin{matrix}
        {\bf f_{rh}}
    \end{matrix}
    \leq
    \begin{bmatrix}
        \begin{smallmatrix}
        150\,N\\
        150\,N\\
        150\,N
        \end{smallmatrix}
    \end{bmatrix}%
    ,\;\;\;\;\;\;
    \begin{bmatrix}
        \begin{smallmatrix}
        -600\,Nm
        \end{smallmatrix}
    \end{bmatrix}
    \leq
    \begin{matrix}
        {\bf\sigma}
    \end{matrix}
    \leq
    \begin{bmatrix}
        \begin{smallmatrix}
        600\,Nm
        \end{smallmatrix}
    \end{bmatrix}
\end{align*}
}}
$$
%
% \begin{center}
% \begin{tabular}{r c l c l}
%   -600.0 & $\leq$ & $f_{lf/rf,x}$ \ $ [N]$ & $\leq$ & 600.0 \\%& -500.0 & \leq \fFs_{0,x} [\rfrac{N}{s}] \leq 500.0 \\
%   -600.0 & $\leq$ & $f_{lf/rf,y}$ \ $ [N]$ & $\leq$ & 600.0 \\%& -500.0 & \leq \fFs_{0,y} [\rfrac{N}{s}] \leq 500.0 \\
%      0.0 & $\leq$ & $f_{lf/rf,z}$ \ $ [N]$ & $\leq$ & 600.0 \\%& -500.0 & \leq \fFs_{0,z} [\rfrac{N}{s}] \leq 500.0 \\
%   -150.0 & $\leq$ & $f_{rh,x}$ \ $ [N]$ & $\leq$ & 150.0 \\%& -500.0 & \leq \fFs_{0,x} [\rfrac{N}{s}] \leq 500.0 \\
%   -150.0 & $\leq$ & $f_{rh,y}$ \ $ [N]$ & $\leq$ & 150.0 \\%& -500.0 & \leq \fFs_{0,y} [\rfrac{N}{s}] \leq 500.0 \\
%   -150.0 & $\leq$ & $f_{rh,z}$ \ $ [N]$ & $\leq$ & 150.0 \\%& -500.0 & \leq \fFs_{0,z} [\rfrac{N}{s}] \leq 500.0 \\
%   -600.0 & $\leq$ & $\sigma_{x}$ \ $[Nm]$ & $\leq$ & 600.0 \\
%   -600.0 & $\leq$ & $\sigma_{y}$ \ $[Nm]$ & $\leq$ & 600.0 \\
%   -600.0 & $\leq$ & $\sigma_{z}$ \ $[Nm]$ & $\leq$ & 600.0 \\
% \end{tabular}
% \end{center}
The bounds are defined from empirical study.
Contact forces can only be applied when a contact is established.
When the contact is released (or moving), we define the lower and upper bounds for the contact force to be zero.
The description of a moving contact is denoted by $\lVert \dot{\pFs}_i \rVert_2 > 0$.
We refer to the contact complementarity to indicate if the end effector is in contact or not given by
\begin{equation}
  \label{eq:contact_complementarity}
  \lVert \dot{\pf} \rVert_2 \cdot \lVert \ff \rVert_2 = 0
\end{equation}
However, note that the complementarity is not explicitly treated yet, but is predefined in the contact configuration.

\subsubsection*{Friction cone constraints}
The applied friction model requires the contact forces to satisfy the so called \textit{friction cone constraints}, which are given for $M$ contact points by
\begin{equation}
  \label{eq:force_friction_cone}
  \|(\ffx, \ffy)\|_2 =\sqrt{(\ffx)^2+ (\ffy)^2} \leq \mu_i \ffz, \; i=1\dots M,
\end{equation}
where $\mu_i>0 $ is the friction coefficient of the contact point $\pf$.
%The friction cone constraints are \textit{second order cone constraints}.
The friction cones $K_1, ..., K_M \subseteq \mathbb{R}^3$ can be defined as
\begin{equation*}
  K_i = \{ {\bf x} \in \mathbb{R}^3 | x^2_1 + x^2_2 \leq (\mu_i x_3)^2, x_3\geq0 \}, \; i= 1\dots M,
  \label{eq:cone_on_norm}
\end{equation*}
and by following this notation, the friction cone constraints of~\eqref{eq:force_friction_cone} can be formulated as
$ \ff \in K_i, \; i=1\dots M $.

\subsubsection*{Kinematic constraints}
For the kinematic feasibility, simplified constraints on the limb lengths relative to the CoM position in global coordinates, denoted by $\com\in\R^3$, are defined by
%\begin{equation}
%  \label{eq:kinematic_constraints}
%  0.64 \leq \lVert \com - \pFs_{lf} \rVert_2 \leq 0.8, \\
%  0.64 \leq \lVert \com - \pFs_{rf} \rVert_2 \leq 0.8
%\end{equation}
\begin{equation}
  \label{eq:kinematic_constraints}
   \underline{\bf L_i} \leq \lVert \com - \pf \rVert_2 \leq \overline{\bf L_i}, \ i = 1,\dots, M.
\end{equation}
We define the leg lengths for $\pFs_{lf}, \pFs_{rf}$ using $\underline{L}_{lf/rf} = 0.64\,m$ and $\overline{L}_{lf/rf} = 0.8\,m$ for the stair climbing motion.


% --- COST ---------------------------------------------------------------------

\subsection{Objective function}\label{sec:cost}
\newcommand{\NOBJ}{4}

Before giving the complete formulation of the optimal-control problem, we first define the cost terms used for the trajectory optimization over a given time interval.
The first term $\ell_0$ keeps the CoM close to the support foot contacts,
$$    \ell_0 = \lVert f_{lf} \rVert_2^2 \cdot \lVert c^{x,y} - p_{lf}^{x,y}  \rVert_2^2
             + \lVert f_{rf} \rVert_2^2 \cdot \lVert c^{x,y} - p_{rf}^{x,y}  \rVert_2^2
             .
    \label{eq:optimal_control_objective_0}
$$
The second term $\ell_1$ uses the complementarity~\eqref{eq:contact_complementarity} to track a reference height depending on the current foot contact height,
$$
    \ell_1 = \lVert (\fFs_{lf,z} + \fFs_{rf,z})(\com_z - \com_z^{ref}) - \fFs_{lf,z} \pFs_{lf,z} - \fFs_{rf,z} \pFs_{rf,z} \rVert_2^2
    .
    \label{eq:optimal_control_objective_1}
$$
The four next terms $\ell_2, \ell_3, \ell_4$ are used to penalize a swaying motion of $\com$ in $z$ direction and stabilize the rotational DoFs.
$$    \ell_2 = \lVert \dcom_z \rVert_2^2
    \label{eq:optimal_control_objective_2}
,\    \ell_3 = \lVert \dcomr_x \rVert_2^2
    \label{eq:optimal_control_objective_3}
,\    \ell_4 = \lVert \dcomr_y \rVert_2^2
    \label{eq:optimal_control_objective_4}
,
  \ell_5 = \lVert \dcomr_z \rVert_2^2
.
    \label{eq:optimal_control_objective_5}
$$
The last term $\ell_6$
%, $\ell_7$
acts as a regularization term,
$$    \ell_6 = \lVert \ddcom \rVert_2^2 + \lVert \ddcomr \rVert_2^2
%, \quad  \ell_7 = \lVert \bf\sigma \rVert_2^2
.
    \label{eq:optimal_control_objective_6}
$$
%
%where objective \eqref{eq:optimal_control_objective_0}
%The objectives \eqref{eq:optimal_control_objective_2}, \eqref{eq:optimal_control_objective_3}, \eqref{eq:optimal_control_objective_4} and \eqref{eq:optimal_control_objective_5} are used to penalize a swaying motion of $\com$ in $z$ direction and stabilize the rotational DoFs.
%This will be utilized together with the free limbs to enable a so called resolved momentum control to stabilize the robot during its motion.
%Objective~\eqref{eq:optimal_control_objective_6}


\begin{table}[ht]
  \vspace*{0.3cm}
  \caption{Objective weights}
  \centering
  \begin{tabular}{|ccccccc|}
    \hline
    $\omega_0$ & $\omega_1$ & $\omega_2$ &
    $\omega_3$ & $\omega_4$ & $\omega_5$ & $\omega_6$ \\
    \hline
    0.05 & 0.0005 & 1.0 & 1.0 & 1.0 & 0.1 & 1.0 \\ \hline
  \end{tabular}
  \label{tab:objective_weights}
\end{table}
