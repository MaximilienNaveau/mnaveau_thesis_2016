This chapter discussed a strategy for a humanoid robot to pull a fire hose while walking towards a desired position and orientation.
%
The main results in this chapter are summarized as follows:
%
%
\begin{list}{ \arabic{point}.}{%
		\usecounter{point}%
		\setlength{\topsep}{5pt}%
		\setlength{\itemsep}{0pt}%
		\setlength{\parsep}{0pt}%
		\setlength{\labelwidth}{3.em}%
		\setlength{\leftmargin}{2em}%
		\setlength{\labelsep}{0.5em}%
	}

\item We proposed a hybrid controller for the robot's wrist holding the fire hose. Position control is used to guarantee no self-collision, while impedance control is employed to pull the hose according to the walking velocity of the robot.

\item Through simulation analysis it was discovered that the hose generates a disturbance on the robot's walking dynamics which in turn produced a drift on the robot's yaw angle.

\item We introduced a walking task to direct the robot to a desired position/orientation and at the same time correct the drift generated when holding the fire hose and walking.

\item We showed experimental results that verified the validity of the proposed controller for pulling the hose and the walking task introduced to correct the walking direction of the robot.

\item We showed that the proposed hybrid controller applied to the wrist holding the hose contributes to the improvement of the robot's balance when walking.

\end{list}
