In order to control the displacement of the humanoid robot HRP-$2$ we used the walking pattern generator presented in Chap.~\ref{chap:nmpc} and published in \cite{naveau:ral:2016}.
As a reminder, this walking pattern generator allows the user to control a humanoid robot almost like a mobile platform.
The input of the walking pattern generator is a velocity $ [\dot{x},\dot{y},\dot{\theta}] $ relative to the ground plane, and is tracked by the center of mass.
The walking pattern generator computes automatically the foot transitions and the center of mass trajectory, so that the robot is balanced when walking.

In this walking pattern generator, the input velocity is tracked by the center of mass if all the constraints are respected.
As a consequence the walking pattern generator will create a swinging motion of the center of mass from one foot to the other even if the input velocity is zero on the robot coronal plane.
The robot will move its center of mass to maintain balance while walking in any circumstances.
%This particular behavior causes perturbations in control laws during the localization of the robot.
%In practice the goal position is set as zone where the CoM has to fitin and 
This swinging motion has to be taken into account when designing the hand tasks to avoid auto-collision and the walking task to make the robot converge toward a specific goal.

