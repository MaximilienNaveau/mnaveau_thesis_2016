\ifdefined\included
\else
\documentclass[a4paper,11pt,twoside]{StyleThese}
\usepackage{amsmath,amssymb}             % AMS Math
\usepackage[french]{babel}
\usepackage[latin1]{inputenc}
\usepackage[T1]{fontenc}
\usepackage{tabularx}
%\usepackage{tabular}
\usepackage{multirow}

\usepackage{hhline}
\usepackage[left=1.5in,right=1.3in,top=1.1in,bottom=1.1in,includefoot,includehead,headheight=13.6pt]{geometry}
\renewcommand{\baselinestretch}{1.05}

% Table of contents for each chapter

\usepackage[nottoc, notlof, notlot]{tocbibind}
\usepackage[french]{minitoc}
\setcounter{minitocdepth}{2}
\mtcindent=15pt
% Use \minitoc where to put a table of contents

\usepackage{aecompl}

% Glossary / list of abbreviations

\usepackage[intoc]{nomencl}
\renewcommand{\nomname}{Liste des Abr�viations}

\makenomenclature

% My pdf code

\usepackage{ifpdf}

\ifpdf
  \usepackage[pdftex]{graphicx}
  \DeclareGraphicsExtensions{.jpg}
  \usepackage[a4paper,pagebackref,hyperindex=true]{hyperref}
  \usepackage{tikz}
  \usetikzlibrary{arrows,shapes,calc}
\else
  \usepackage{graphicx}
  \DeclareGraphicsExtensions{.ps,.eps}
  \usepackage[a4paper,dvipdfm,pagebackref,hyperindex=true]{hyperref}
\fi

\graphicspath{{.}{images/}}

%nicer backref links
\renewcommand*{\backref}[1]{}
\renewcommand*{\backrefalt}[4]{%
\ifcase #1 %
(Non cit�.)%
\or
(Cit� en page~#2.)%
\else
(Cit� en pages~#2.)%
\fi}
\renewcommand*{\backrefsep}{, }
\renewcommand*{\backreftwosep}{ et~}
\renewcommand*{\backreflastsep}{ et~}

% Links in pdf
\usepackage{color}
\definecolor{linkcol}{rgb}{0,0,0.4} 
\definecolor{citecol}{rgb}{0.5,0,0} 
\definecolor{linkcol}{rgb}{0,0,0} 
\definecolor{citecol}{rgb}{0,0,0}
% Change this to change the informations included in the pdf file

\hypersetup
{
bookmarksopen=true,
pdftitle="�valuation de la s�curit� des �quipements grand public connect�s � Internet",
pdfauthor="Yann BACHY", %auteur du document
pdfsubject="Th�se", %sujet du document
%pdftoolbar=false, %barre d'outils non visible
pdfmenubar=true, %barre de menu visible
pdfhighlight=/O, %effet d'un clic sur un lien hypertexte
colorlinks=true, %couleurs sur les liens hypertextes
pdfpagemode=None, %aucun mode de page
pdfpagelayout=SinglePage, %ouverture en simple page
pdffitwindow=true, %pages ouvertes entierement dans toute la fenetre
linkcolor=linkcol, %couleur des liens hypertextes internes
citecolor=citecol, %couleur des liens pour les citations
urlcolor=linkcol %couleur des liens pour les url
}

% definitions.
% -------------------

\setcounter{secnumdepth}{3}
\setcounter{tocdepth}{2}

% Some useful commands and shortcut for maths:  partial derivative and stuff

\newcommand{\pd}[2]{\frac{\partial #1}{\partial #2}}
\def\abs{\operatorname{abs}}
\def\argmax{\operatornamewithlimits{arg\,max}}
\def\argmin{\operatornamewithlimits{arg\,min}}
\def\diag{\operatorname{Diag}}
\newcommand{\eqRef}[1]{(\ref{#1})}

\usepackage{rotating}                    % Sideways of figures & tables
%\usepackage{bibunits}
%\usepackage[sectionbib]{chapterbib}          % Cross-reference package (Natural BiB)
%\usepackage{natbib}                  % Put References at the end of each chapter
                                         % Do not put 'sectionbib' option here.
                                         % Sectionbib option in 'natbib' will do.
\usepackage{fancyhdr}                    % Fancy Header and Footer

% \usepackage{txfonts}                     % Public Times New Roman text & math font
  
%%% Fancy Header %%%%%%%%%%%%%%%%%%%%%%%%%%%%%%%%%%%%%%%%%%%%%%%%%%%%%%%%%%%%%%%%%%
% Fancy Header Style Options

\pagestyle{fancy}                       % Sets fancy header and footer
\fancyfoot{}                            % Delete current footer settings

%\renewcommand{\chaptermark}[1]{         % Lower Case Chapter marker style
%  \markboth{\chaptername\ \thechapter.\ #1}}{}} %

%\renewcommand{\sectionmark}[1]{         % Lower case Section marker style
%  \markright{\thesection.\ #1}}         %

\fancyhead[LE,RO]{\bfseries\thepage}    % Page number (boldface) in left on even
% pages and right on odd pages
\fancyhead[RE]{\bfseries\nouppercase{\leftmark}}      % Chapter in the right on even pages
\fancyhead[LO]{\bfseries\nouppercase{\rightmark}}     % Section in the left on odd pages

\let\headruleORIG\headrule
\renewcommand{\headrule}{\color{black} \headruleORIG}
\renewcommand{\headrulewidth}{1.0pt}
\usepackage{colortbl}
\arrayrulecolor{black}

\fancypagestyle{plain}{
  \fancyhead{}
  \fancyfoot{}
  \renewcommand{\headrulewidth}{0pt}
}

%\usepackage{MyAlgorithm}
%\usepackage[noend]{MyAlgorithmic}
\usepackage[ED=MITT - STICRT, Ets=INSA]{tlsflyleaf}
%%% Clear Header %%%%%%%%%%%%%%%%%%%%%%%%%%%%%%%%%%%%%%%%%%%%%%%%%%%%%%%%%%%%%%%%%%
% Clear Header Style on the Last Empty Odd pages
\makeatletter

\def\cleardoublepage{\clearpage\if@twoside \ifodd\c@page\else%
  \hbox{}%
  \thispagestyle{empty}%              % Empty header styles
  \newpage%
  \if@twocolumn\hbox{}\newpage\fi\fi\fi}

\makeatother
 
%%%%%%%%%%%%%%%%%%%%%%%%%%%%%%%%%%%%%%%%%%%%%%%%%%%%%%%%%%%%%%%%%%%%%%%%%%%%%%% 
% Prints your review date and 'Draft Version' (From Josullvn, CS, CMU)
\newcommand{\reviewtimetoday}[2]{\special{!userdict begin
    /bop-hook{gsave 20 710 translate 45 rotate 0.8 setgray
      /Times-Roman findfont 12 scalefont setfont 0 0   moveto (#1) show
      0 -12 moveto (#2) show grestore}def end}}
% You can turn on or off this option.
% \reviewtimetoday{\today}{Draft Version}
%%%%%%%%%%%%%%%%%%%%%%%%%%%%%%%%%%%%%%%%%%%%%%%%%%%%%%%%%%%%%%%%%%%%%%%%%%%%%%% 

\newenvironment{maxime}[1]
{
\vspace*{0cm}
\hfill
\begin{minipage}{0.5\textwidth}%
%\rule[0.5ex]{\textwidth}{0.1mm}\\%
\hrulefill $\:$ {\bf #1}\\
%\vspace*{-0.25cm}
\it 
}%
{%

\hrulefill
\vspace*{0.5cm}%
\end{minipage}
}

\let\minitocORIG\minitoc
\renewcommand{\minitoc}{\minitocORIG \vspace{1.5em}}

\usepackage{multirow}
%\usepackage{slashbox}

\newenvironment{bulletList}%
{ \begin{list}%
	{$\bullet$}%
	{\setlength{\labelwidth}{25pt}%
	 \setlength{\leftmargin}{30pt}%
	 \setlength{\itemsep}{\parsep}}}%
{ \end{list} }

\newtheorem{definition}{D�finition}
\renewcommand{\epsilon}{\varepsilon}

% centered page environment

\newenvironment{vcenterpage}
{\newpage\vspace*{\fill}\thispagestyle{empty}\renewcommand{\headrulewidth}{0pt}}
{\vspace*{\fill}}

\usepackage{tablefootnote}
\newcommand{\widthValue}{0.13\linewidth}
\sloppy
\begin{document}
\fi

%\printnomenclature
\cleardoublepage
\begin{vcenterpage}
%\small

%\vspace*{-4.5cm}
\noindent\rule[2pt]{\textwidth}{0.5pt}
\\
{\large\textbf{R\'esum\'e en fran\c cais:}}

%Cette th\`ese traite du probl\`eme de la locomotion des robots humano\"ides.
%Elle a \'et\'e r\'ealis\'ee en collaboration entre les partenaires du projet europ\'een \koroibot.
%L'objectif de ce projet est l'am\'elioration des capacit\'es des robots humano\"ides \`a se mouvoir, comme les humains, de fa\c con dynamique et polyvalente.
%Les travaux de recherche et innovation au sein du projet portent principalement sur le d\'eveloppement de nouveaux algorithmes de g\'en\'eration de mouvement pour des robots humano\"ides d\'ej\`a existants.
%Il s'attaque aussi \`a la conception de la future g\'en\'eration de robots humano\"ides.
%
%Les contributions majeures de cette th\`ese reposent sur la conception de nouveaux algorithmes temps-r\'eel de contr\^ole pour la locomotion des robots  humano\"ides et leur int\'egration sur le robot HRP-2.
%Deux contr\^oleurs seront pr\'esent\'es, le premier permettant la locomotion multi-contacts avec une connaissance a priori des futures positions des contacts.
%Le deuxi\`eme \'etant une extension d'un travail r\'ealis\'e sur de la marche sur sol plat am\'eliorant les performances et ajoutant des fonctionnalit\'ees au pr\'ec\'edent algorithme.
%De plus, des contributions techniques seront mises en \'evidence durant les multiples exp\'eriences effectu\'ees pendant la th\`ese et dans le contexte du projet \koroibot.
%En effet, au sein du projet, tous les algorithmes de g\'en\'eration de mouvement sont test\'es sur plusieurs sc\'enarios.
%Des indices de performance ont \'et\'e calcul\'es au d\'ebut et \`a la fin du projet.
%Au LAAS-CNRS nous avons choisi comme d\'elivrable de faire marcher HRP-2 sur un sol plat, sur une poutre, dans des escaliers, sur un guet et de r\'esister \`a des perturbations ext\'erieures.
%En collaborant avec des sp\'ecialistes du mouvement humain nous avons implement\'e un contr\^oleur innovant permettant de suivre des trajectoires cycliques du centre de masse.
%Nous pr\'esenterons aussi un contr\^oleur corps-complet utilisant, pour le haut du corps, des primitives de mouvements extraites du mouvement humain et pour le bas du corps, un g\'en\'erateur de marche.
%Les r\'esultats de cette th\`ese ont \'et\'e int\'egr\'es dans la suite logicielle "Stack-of-Tasks" du LAAS-CNRS.

Cette th\`ese traite du probl\`eme de la locomotion des robots humano\"ides
dans le contexte du projet europ\'een \koroibot.
En s'inspirant de l'\^etre humain, l'objectif de ce projet est
l'am\'elioration des capacit\'es des robots humano\"ides \`a se mouvoir de fa�on
dynamique et polyvalente.
Le coeur de l'approche scientifique repose sur l'utilisation du controle
optimal, \`a la fois pour l'identification des couts optimis\'es par l'\^etre
humain et pour leur mise en oeuvre
sur les robots des partenaires roboticiens. Cette th\`ese s'illustre donc
par une collaboration \`a la fois avec des math\'ematiciens du contr�le et
des sp\'ecialistes de la mod\'elisation
des primitives motrices.

Les contributions majeures de cette th\`ese reposent donc sur la
conception de nouveaux algorithmes temps-r\'eel de contr�le pour la
locomotion des robots humano\"ides avec nos coll\'egues de l'universit\'e
d'Heidelberg et leur int\'egration sur le robot HRP-2. Deux contr�leurs
seront pr\'esent\'es, le premier permettant la locomotion multi-contacts
avec une connaissance a priori des futures positions des contacts. Le
deuxi\`eme \'etant une extension d'un travail r\'ealis\'e sur de la marche sur
sol plat am\'eliorant les performances et ajoutant des fonctionnalit\'ees au
pr\'ec\'edent algorithme.  En collaborant avec des sp\'ecialistes du mouvement
humain nous avons implement\'e un contr�leur innovant permettant de suivre
des trajectoires cycliques du centre de masse. Nous pr\'esenterons aussi
un contr�leur corps-complet utilisant, pour le haut du corps, des
primitives de mouvements extraites du mouvement humain et pour le bas du
corps, un g\'en\'erateur de marche. Les r\'esultats de cette th\`ese ont \'et\'e
int\'egr\'es dans la suite logicielle "Stack-of-Tasks" du LAAS-CNRS.

{\large\textbf{Mots cl\'es:}}
Locomotion bip\`ede et humano\"ides, Robots humano\"ides, Commande pr\'edictive non lin\'eaire

\noindent\rule[2pt]{\textwidth}{0.5pt}
\\
\noindent{\large\textbf{Abstract:}}

This thesis covers the topic of humanoid robot locomotion in the frame of the European project \koroibot.
The goal of this project is to enhance the ability of humanoid robots to walk in a dynamic and versatile fashion as humans do.
Research and innovation studies in \koroibot\ rely on optimal control methods both for the identification of cost functions used by human being and for their implementations on robots owned by roboticist partners.
Hence, this thesis includes fruitful collaborations with both control mathematicians and experts in motion primitive modeling.

The main contributions of this PhD thesis lies in the design of new real time controllers for humanoid robot locomotion with our partners from the University of Heidelberg and their integration on the HRP-2 robot.
Two controllers will be shown, one allowing multi-contact locomotion with a prior knowledge of the future contacts.
And the second is an extension of a previous work improving performance and providing additional functionalities.
In a collaboration with experts in human motion we designed an innovating controller for tracking cyclic trajectories of the center of mass.
We also show a whole body controller using upper body movement primitives extracted from human behavior and lower body movement computed by a walking pattern generator.
The results of this thesis have been integrated into the LAAS-CNRS "Stack-of-Tasks" software suit.

{\large\textbf{Key words:}}
Humanoid and Bipedal Locomotion, Humanoid Robots, Nonlinear Model Predictive Control

\end{vcenterpage}
\ifdefined\included
\else
\end{document}
\fi